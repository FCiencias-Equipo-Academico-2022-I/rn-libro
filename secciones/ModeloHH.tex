
\section{Introducción}
Esta sección se enfocará a la parte de transmisión de información y que tipo de operaciones lógicas matemáticas ocurren para que un cerebro pueda realizar cómputos, específicamente se detallará la mecánica de los disparos de las neuronas, siendo estos una de las características más relevantes a la hora de modelar las redes neuronales artificiales. Si en algún momento de su vida han visto temas relacionados con compuertas digitales, arquitectura de computadoras, diseño electrónico digital, les será más fácil abstraer el concepto, pues nosotros vamos a ver los procesos de paso de información a través de compuertas pero en un sistema biológico (de la naturaleza). 

Notemos primeramente un impulso nervioso, recordemos que esté es una onda que avanza desde el cono axónico de la neurona hasta la neurona postsináptica. Esta onda electroquímica ocurre dada la diferencia de potencial entre la parte interna y externa de neurona, está diferencia se da a consecuencia de las distintas concentraciones de iones en ambos lados de la membrana plasmática. Los estados en la membrana plasmática (del axón) se pueden diferenciar en, potenciales neuronales:

\begin{itemize}
\item \textbf{Potencial de reposo:} Es la diferencia de cargas en la membrana y está polarizada a -70mV. Es positiva por fuera (Na+ ) y negativa por dentro por Cl- y proteínas- y no transmite señal. 
\item \textbf{Potencial de acción o membrana:} Un estímulo umbral de 55mV, despolariza la membrana y abre los canales del Na+ y K+ y avanza la señal nerviosa, es un cambio muy rápido en la polaridad de la membrana de negativo a positivo y vuelta a negativo.
\end{itemize}

%(Insertar esquema) 
\begin{figure}[h]
 \centering
 \includegraphics[scale=0.5]{../Figuras/Grafica.png}
 \caption{Representación gráfica de la respuesta de los canales ionicos de sódio (Na+ en verde) y potasio (K+ en azul) ante un estímulo de voltaje, dando como resultado un potencial de acción que viajará a lo largo de todo el axón.}
 \label{fig:graficaP}
\end{figure}

Retomando la sinapsis eléctrica, donde participan los canales iónicos y las entradas de la neuronas (dendritas) están siendo alteradas poco a poco, hasta que ocurre la suficiente carga (diferencia de potencial) en sus dendritas y en el cuerpo de la neurona, para que desde el cono axónico se de un disparo o potencial de acción (spike), transmitiendo la información gracias a la apertura y cierre de ciertos canales de iones cargados. Este cambio brusco de la diferencia de potencial, se nota en forma de un pulso eléctrico (ver \ref{fig:graficaP}),  para saber más a detalle qué está ocurriendo en está rápida elevación en la diferencia de potencial, se contará de dónde salió este modelo y por qué toma la forma que tiene. 

Los primeros científicos que estudiaron el potencial de acción y dieron un modelo (de la unión sináptica eléctrica) fueron Alan Lloyd Hodgkin y Andrew Fielding Huxley alrededor de 1952, obteniendo un modelo matemático \footnote{El texto original de este experimento se puede encontrar en la siguiente url: \url{ https://physoc.onlinelibrary.wiley.com/doi/pdf/10.1113/jphysiol.1952.sp004764}}, que intenta explicar qué es lo que estaba pasando en las neuronas. Ellos trabajaron con un calamar gigante (que puede medir hasta 4 metros de largo) que dado su gran tamaño, tiene un axón también bastante gigantesco, que recorre casi la mitad del cuerpo del calamar y su grosor es de medio milímetro, considerando el tamaño estándar de una axón de una neurona (1-20 µm). El axón del calamar gigante es tan grande que les permitió introducir dispositivos para medir el voltaje, es decir, la diferencia de potencial entre, el interior de la neurona y la parte de afuera (el ambiente externo de la neurona). Con estás mediciones experimentales que lograron obtener, se pudo determinar qué pasaba con las cargas eléctricas tanto en el interior como en el exterior y así estudiar cómo se lograba la transferencia de electricidad cuando disparaba este pulso. 
 Se dan cuenta que podían modelar este comportamiento como un circuito eléctrico donde están corriendo estas corrientes, si bien aún no sabían todavía cuál era exactamente el mecanismo biológico por detrás, si observaron que había dos elementos protagónicos que serían el sodio y el potasio.
 Notaron que estos existen en diferentes concentraciones, en la parte de afuera y en la parte de adentro de las neuronas. Con esto nosotros podemos aprender también el por qué es importante consumir algo de sal y nunca estar bajos de potasio, pues estos dos elementos son indispensables para que las neuronas puedan transmitir sus señales. 

\section{Membrana y canal}

Hodgkin y Huxley se dedicaron a estudiar qué pasaba con las concentraciones de estos iones (sodio y potasio) en la parte de afuera o en la parte de adentro cuando empezaban a fluir las corrientes. El sistema parecía una especie de circuito eléctrico, se lo imaginaron como una especie de membrana porosa (lo cual es bastante cercano a lo que después se descubrió  con la microscopía) y la forma en que lo vieron fue como un circuito eléctrico donde  \textit{la membrana está funcionando como un capacitor} que almacena ligeramente las cargas cuando están tratando de pasar de un lado hacia el otro y además con la cualidad que tenía de veces dejar pasar más iones y a veces no (semipermeable), modelan esto como una especie de \textit{resistencias variables}. Bajo ciertas condiciones de voltaje de la diferencia de potencial entre la parte de afuera y la parte de adentro, estos canales permiten pasar más de estos iones (ya sean sodio, potasio o calcio) o por el contrario impiden su paso (ver \ref{fig:ModelHh}).


\begin{figure}[h]
 \centering
 \includegraphics[scale=0.5]{../Figuras/ModeloHH.2}
 \caption{Un primer modelo de la membrana axónica modelada como circuito eléctrico. La parte amarilla es la membrana}
 \label{fig:ModelHh}
\end{figure}


Ahora se necesitan más detalles de la representación de los canales y toman en cuenta que el comportamiento de estas resistencias viene acompañado con un voltaje de reposo, en estos voltajes particulares cada tipo de ion (de la resistencia modelada) se estabiliza y ya no va a cambiar esta resistencia (ver \ref{fig:circuito}). 


\begin{figure}[h]
 \centering
 \includegraphics[scale=0.5]{../Figuras/circuito.png}
 \caption{Modelo de la membrana axónica modelada como circuito eléctrico, con los distintos canales presentes y su voltaje de reposo.}
 \label{fig:circuito}
\end{figure}

Lo que observan es que el \textbf{ion de sodio (Na+)} y su resistencia va a variar dependiendo del voltaje, a esto se le llama un \textbf{canal transitorio} porque en ciertos voltajes si puede pasar; si es muy bajo, no puede pasar y si rebasa un cierto umbral entonces se vuelve a tapar y ya no puede pasar. 
Lo que sucede con el \textbf{ion de potasio (K+)} es que, puede salir si el voltaje está más allá de un cierto valor, si no, no pasan y va variando un poquito que tanto puede pasar, a esto se le llama \textbf{canal persistente}. Por estas características de que el potasio es un intervalo dentro de la recta y el sodio es a partir de cierto valor, por tanto se les modelan de maneras ligeramente diferentes. Más adelante se descubrió porque tenían este comportamiento, básicamente el canal de potasio es una puerta hecha de cuatro subpuertas por donde los elementos pasan o no pasan, el canal de sodio es como una compuerta que está hecha de tres subpuertas que se pueden abrir y tiene aparte un tapón extra, que hace que  aunque estas tres están abiertas bloquee toda la compuerta.
Las neuronas están trabajando con muchos más iones aparte de estos dos, uno que destaca bastante es el caso del cloro (Cl-) que tiene carga negativa. Se tienen canales para intercambio aleatorio de otros iones, \textbf{L} un canal aleatorio (leaky).

Entonces con lo que ellos midieron experimentalmente, midieron cómo se estaban comportando estas resistencias dependiendo del voltaje o la diferencia de potencial que había entre ambos lados de la membrana y a partir de ahí pudieron describir matemáticamente y simular los disparos que se conocen como potenciales de acción vamos a ver cuáles fueron estos conceptos de electricidad que se están utilizando para el modelo tenemos este concepto de potenciales eléctricos.

\begin{itemize}
\item Potenciales eléctricos \emph{E ó  V}; resultan de la separación de cargas opuestas. Se mide en \emph{mV}.
    \begin{itemize}
     \item \emph{E\textsubscript{(Na,K,L)}} voltaje en reposo para los iones de Na, K y L, o también conocido como  potencial de inversión iónico, es el potencial de membrana en el que no hay flujo neto (total) de ese ion en particular de un lado de la membrana al otro. 
     \item \emph{V\textsubscript{m}}, el potencial eléctrico de la membrana. 
     \end{itemize}

\item Corriente \emph{I}; Movimiento de cargas. Se mide en \emph{µA}.
    \begin{itemize}
     \item \emph{I\textsubscript{(Na,K,L)}} corriente entrante a los canales de Na, K o L.
     \end{itemize}

\item Resistencia \emph{R}; Medida de la oposición al movimiento de las partículas cargadas.
\item Capacitancia o capacidad eléctrica \emph{C} . Cantidad de energía eléctrica almacenada en un capacitor para una diferencia de potencial eléctrico dada.
    \begin{itemize}
     \item \emph{C\textsubscript{m}} la capacitancia de la membrana. 
     \end{itemize}

\item Conductancia \emph{g}; Inverso de la resistencia \( \dfrac{1}{R} \) , es decir, facilidad de transmisión de las partículas cargadas.
    \begin{itemize}
     \item \emph{g\textsubscript{(Na,K,L}}  la conductancia del canal de sodio, potasio, cloro y la L también refiriendose a otros canales de iones. 
     \end{itemize}

\end{itemize}

Lo que está pasando en los \textbf{potenciales eléctricos} es que hay mucho sodio en la parte externa de la membrana por ej. tres iones de sodio que son cargas positiva por dos iones de potasio que hay en la parte interna, entonces hay muchas más cargas positivas en la parte de afuera que las que hay en la parte de adentro y eso es lo que provoca  la diferencia de cargas que es lo que estamos viendo como un  potencial eléctrico.

La  capacidad eléctrica o \textbf{capacitancia} es la que estamos utilizando para modelar la membrana conformada por lípidos, que es una  capa de grasa y esa es la cantidad de energía eléctrica almacenada en un capacitor para una diferencia de potencial eléctrico dada.Éste comportamiento bastante interesante porque las cargas quedan almacenadas un momento pero se van liberando poco a poco y se va descargando ese capacitor. 

Durante el experimento con el axón, se le dierón cargas electricas directamente al axón y gracias a eso lograban ir midiendo que era lo que estaba pasando con las concentraciones de cargas afuera y adentro en el caso de las neuronas reales, esto en un ambiente no alterado ocurre cuando entran en juego los neurotransmisores y provocan que haya cambios, en estas corrientes. Entonces hodgkin y huxley  jugaron el rol que tendrían que jugar usualmente los \emph{neurotransmisores} para abrir otras compuertas. Nosotros en la manera en la que lo vamos a simular es precisamente con estas corrientes que son las que se están poniendo en el experimento y vamos a ver cómo reacciona el axón. 


\section{Potenciales de Nerst o de reposo}

Son los potenciales a los cuales el flujo neto de iones a través de los canales abiertos es cero.
Aquí vemos precisamente porque estamos utilizando la \emph{E} generalmente la vamos a utilizar para referirnos a la diferencia de potencial entre la parte de afuera de la célula y la parte de adentro  las vamos a utilizar para representar a aquellos voltajes donde cada una de las compuertas encontrarían su equilibrio . Esos voltajes son distintos para cada una de las compuertas, esto va a provocar precisamente la dinámica de la de la neurona, por ejemplo: 

\begin{itemize}
\item E \textsubscript{Na}  \emph{50mV}
\item E \textsubscript{Ca}  \emph{150mV}
\item E \textsubscript{K}   \emph{− 80mV}
\item E \textsubscript{Cl}  \emph{− 60mV}
\end{itemize}

Aquí vemos que el sodio estaría su equilibrio en un valor positivo, 
el calcio que es el que va a jugar un rol de que se activen los neurotransmisores y se transmita el disparo, observamos que el voltaje tendría que ser bastante positivo. El potasio que es el que usualmente está trabajando intercambiándose casi todo el tiempo en la neurona, veremos que el punto de equilibrio usual de la neurona anda por los -76mV y el del cloro, cada uno de estos canales pues está tratando de jalar la dinámica hacia su potencial de equilibrio y no hay precisamente un acuerdo entre ellos y eso es precisamente lo que hace que las neuronas cobren "vida".

\section{Modelo de la membrana como bicapa de lípidos}
\hypertarget{LaEq}{La membrana} de una neurona es modelada como un elemento de un circuito con capacitancia \emph{C\textsubscript{m}} y potencial \emph{V} ,las corriente que fluye a través de la bicapa lipída están regidos por las siguientes ecuaciones:

\begin{equation}
  I_{m} = C_{m} \dfrac{dV_{m}}{dt}
  \label{eq:corrientesEnLaMembrana}
\end{equation}

Está sería la ecuación principal (\ref{eq:corrientesEnLaMembrana}) donde \(\dfrac{dVm}{dt}\) está representando el cambio voltaje en la membrana respecto al tiempo.

\begin{equation}
  C_{m} \dfrac{dV_{m}}{dt} =  - g_{Na} m^3 h(V - E_{Na} ) - g_{K} n 4 (V - E_{K} ) - g_{L} (V - E_{L} ) + I_ext
  \label{eq:corrientesEnLaMembrana2}
\end{equation}

Cada una de las partes del lado izquierdo de la ecuación \ref{eq:corrientesEnLaMembrana2} corresponde a las compuertas de los canales y la corriente de un estimulo externo que pueda influir a la membrana (este estimulo siempre será desde al exterior hacia el interior).

Retomando lo escrito anteriormente el canal de sodio es una compuerta compuesta de \textbf{tres} subpuertas y una subpuerta que actua como tapón y el canal de potasio es una compuerta compuesta de \textbf{cuatro} subpuertas iguales, \hypertarget{secc} {con esto podemos notar claramente que las conductancias sean representadas como}:

\begin{itemize}
 \item \(\dfrac{1}{R_{Na}} = g_{Na} * m ^3 * h \) donde \(g_{Na}\) es una constante que representa el valor de la conductancia maxima, \textbf{m} es la proporción de los canales de sodio abiertos (representa la concentración de sodio) y nos indica la activación (subpuertas abiertas) del canal, \textbf{h} es el “tapón” de la compuerta que puede impedir el paso de iones independientemente de las otras tres subpuertas, es decir la inactivación (compuerta bloqueda).
Los movimientos combinados de \textbf{m} y \textbf{h} son los que controlan la compuerta de sodio.
 \item \(\dfrac{1}{R_{K}} = g_{K} * n^4\) donde \(g_{K}\) es una constante que representa el valor de la conductancia maxima, \textbf{n} es la proporción de los canales de potasio abiertos (representa la concentración de potasio) y nos indica la activación del canal de potasio.
 \item \(g_{L}\) es una constante, de los canales por fuga, que representa la concentración de los demás iones que pasan por la membrana.
\end{itemize}

Ahora \emph{m}, \emph{n} y \emph{h}, son variables de activación que describen la probabilidad de que los canales iónicos estén abiertos, se puede describir mediante las siguientes ecuaciones diferenciales ordinarias:

\begin{equation}
  \dfrac{1}{\gamma(T)}\dfrac{dn}{dt} =  \alpha_{n^\infty} (V)(1 - n) - \beta_{n} (V) n = \dfrac{n(V)-n(t)}{\tau_{n}(V)}
  \label{eq:corrientesEnLaMembrana3}
\end{equation}

\begin{equation}
  \dfrac{1}{\gamma(T)}\dfrac{dm}{dt} =  \alpha_{m} (V)(1 - m) - \beta_{m} (V) m = \dfrac{m^\infty(V)-m(t)}{\tau_{m}(V)}
  \label{eq:corrientesEnLaMembrana4}
\end{equation}

\begin{equation}
  \dfrac{1}{\gamma(T)}\dfrac{dh}{dt} =  \alpha_{h} (V)(1 - h) - \beta_{h} (V) h = \dfrac{h^\infty(V)-h(t)}{\tau_{h}(V)}
  \label{eq:corrientesEnLaMembrana5}
\end{equation}

donde la ecuación \ref{eq:corrientesEnLaMembrana3} representa al canal de potasio y las ecuaciones \ref{eq:corrientesEnLaMembrana4} y \ref{eq:corrientesEnLaMembrana5} representando al canal de sodio tomando en cuenta que tiene dos tipos de subpuertas.

Las expresiones de \(\alpha\) y \(\beta\) estan dadas por las siguientes ecuaciones:

\begin{align*}
\alpha_{n}&=\dfrac{0.01(10-V)}{exp(\dfrac{10-V}{10})-1}           &  \beta_{n}&=0.125exp-\dfrac{V}{80}\\
\alpha_{m}&=\dfrac{0.01(25-V)}{exp(\dfrac{25-V}{10})-1}                    &  \beta_{m}&=4exp-\dfrac{V}{18}\\
\alpha_{h}&=\dfrac{0.07}{exp-(\dfrac{V}{20})}              &  \beta_{h}&=\dfrac{1}{1+exp\dfrac{30-V}{10}}
\end{align*}

Los factores \(\alpha\) y \(\beta\) se denominan como constantes de velocidad de transición. \(\alpha\) es el número de veces por segundo que se abre una puerta que está en estado cerrado, mientras que \(\beta\) es el número de veces por segundo que se cierra una puerta que está en estado abierto. Si la membrana tiene un la carga negativa, \(\alpha\) debe aumentar y la \(\beta\) debe disminuir, cuando la membrana este despolarizada.

Hasta ahora sabemos que en la bicapa de lipidos, una pequeña carga está pasando entre sus capas de grasa. También sabemos que la carga es almacenada por un breve periodo de tiempo, dando como resultado que la bicapa se comporte como un \textbf{capacitor}. Esta membrana también está con cierta resistencia al paso de corriente. Con esto tenemos el siguiente diagrama \footnote{Otra explicación más profunda de las ecuaciones dadas partir del diagrana \ref{fig:circuitoP}la podemos encontrar en \url{https://neurowiki.case.edu/wiki/Action_Potential_IV:_Hodgkin-Huxley_Equations_and_Other_Conductances}} \ref{fig:circuitoP}

\begin{figure}[h]
 \centering
 \includegraphics[scale=0.5]{../Figuras/bicapaLipidos.png}
 \caption{Modelo de la bicapa de lipidos donde, V son los cambios de voltaje en la membrana que es el potencial eléctrico,\(I_{C}\) es la corriente del capacitor, \(I_{R}\) es la corriente de la resistencia, \(C_{m}\) es la capacitancia de la membrana, R es la resistencia}
 \label{fig:circuitoP}
\end{figure}

Tenemos dadas las siguientes ecuaciones:
\begin{equation}
    I_{C} + I_{R} - I_{ext} = 0
  \label{eq:bicapa1}
\end{equation}

\begin{equation}
    C\dfrac{dV}{dt} + \dfrac{V}{R} - I_{ext} = 0 \\
  \label{eq:bicapa2}
\end{equation}

\begin{equation*}
    C\dfrac{dV}{dt} = -\dfrac{V}{R} + I_{ext} 
\end{equation*}

Ahora por la ley de corriente de Kirchhoff \footnote{La ley de la corriente de Kirchhoff dice que la suma de todas las corrientes que fluyen hacia un nodo es igual a la suma de las corrientes que salen del nodo} tenemos que la suma de las corrientes del capacitor y la recistencia debe ser cero
por la concervación de corriente y si consideramos un factor adicional de una corriente externa aplicada o administrada a la neurona, tenemos la ecuación \ref{eq:bicapa1}.

Después tenemos la relación entre la diferencia de potenciales, que almacena energía y la carga eléctrica que guarda, donde: \emph{C} es la capacidad, medida en faradios, \emph{Q} la carga eléctrica almacenada, medida en culombios, \emph{V} la diferencia de potencial medida en voltios. Entonces \(C = Q/V\), despejando a \emph{Q} tenemos \emph{Q = CV} y derivando de ambos lados respecto al tiempo y conciderando que \emph{C} es una constante al ser una propiedad de la membrana \(\dfrac{dQ}{dt} = C\dfrac{dV}{dt}\). Como la definición de corriente es el cambio de carga en el tiempo tenemos que \(I_{C} = C\dfrac{dV}{dt}\). Notemos finalmente la corriente de la resistencia \(I_{R}\), recordando la ley de Ohm \footnote{La ley de Ohm establece que la diferencia de potencial V que aplicamos entre los extremos de un conductor determinado es directamente proporcional a la intensidad de la corriente I que circula por el conductor, es decir \(V = R * I\). Notemos también que \(V = V_{m} - V_{rest}\) } la podemos rescribir como \(\dfrac{V-V_{rest}}{R}\). Sustituyendo de lo anterior en la ecuación \ref{eq:bicapa1} se obtiene la siguiente ecuación:

\begin{equation}
 C\dfrac{dV}{dt} + \dfrac{V-V_{rest}}{R} - I_{ext} = 0
 \label{eq:bicapa3}
\end{equation}

\begin{equation*}
 C\dfrac{dV}{dt} = -\dfrac{V-V_{rest}}{R} + I_{ext} 
\end{equation*}

Ahora multiplicando todo por R:
\begin{equation}
 RC\dfrac{dV}{dt} = -V + (V_{rest} + RI_{ext}) 
 \label{eq:bicapa4}
\end{equation}

Denotando RC como la constante de tiempo \(\tau\) y tomando en cuenta que en cuanto se aplica la corriente va a empezar a cambiar el voltaje poco a poco hasta establecerse en un voltaje de equilibrio (ahí se va a quedar quieta). Entonces cuando el voltaje ya no está cambiando con el tiempo quiere decir que su derivada con respecto al tiempo es cero. Observemos que \(dV/dt = 0\) significaría que V es igual a infinito y que el voltaje en el estado estacionario cuando , \(dV/dt = 0\) depende del potencial de reposo y del producto entre la resistencia con la corriente externa suministrada, entonces tenemos que:

\begin{equation}
 V_{\infty} = V_{rest} + RI_{ext}
 \label{eq:bicapa5}
\end{equation}

Sustituyendo con \ref{eq:bicapa5} y la constante, en \ref{eq:bicapa4} tenemos que:
\begin{equation}
 \tau\dfrac{dV}{dt} = -V + V_{\infty}
 \label{eq:bicapa6}
\end{equation}

\subsection{Las conductancias ionicas}
Nuestro objetivo aquí es encontrar ecuaciones que describan las conductancias con precisión razonable y lo suficientemente simples para el cálculo teórico de \emph{el potencial de acción} y \emph{el período refractario}. 

Si tomamos las ecuaciones diferenciales anteriores notamos que las soluciones tienen este tipo de forma \ref{fig:graficaX}:

\begin{figure}[h]
 \centering
 \includegraphics[scale=0.8]{../Figuras/solPulso1.png}
 \caption{Soluciones para el pulso.}
 \label{fig:graficaX}
\end{figure}


Donde si estamos aplicando una corriente externa lo que sucede es lo que estamos viendo en azul, un exponencial que va creciendo y que tiende hacia un cierto valor límite que sería de infinito. Si dejamos de aplicar la corriente externa entonces ahora tendremos un exponencial pero que tiende hacia el cero y se va a estabilizar en cero, lo que vemos en rojo. 

\begin{figure}[h]
 \centering
 \includegraphics[scale=0.5]{../Figuras/solPulso2.png}
 \caption{Soluciones para el pulso escalón.}
 \label{fig:graficaX1}
\end{figure}

Simulando lo que hicieron Hodgkin y Huxley que fue al axón de repente darle un toque, siendo en el origen de la gráfica (que visualizamos en la figura \ref{fig:graficaX1}) la parte en la que le están dando el toque al axón, momentos antes estaba quieta la neurona de repente le aplican una cierta cantidad de electricidad y va a empezar a cambiar el comportamiento de los canales la porosidad de la membrana, vamos a ver que empieza a incrementarse la diferencia de potencial hasta que llegan a un nuevo equilibrio (alrededor de t = 3) y si siguieran dándole el toque en esa cantidad pues ya se quedaría ahí la neurona ya no veríamos más cambios lo que va a suceder entonces es que, retiramos las pinzas (se le deja de dar el toque) y los canales otra vez van a empezar a regresar a la normalidad y vamos a ver un descenso en adelante.

Entonces hasta aquí ya tenemos la idea de cómo va a reaccionar la neurona ante cierto estimulo, sin embargo esto que acabamos de ver en las gráficas sería como si tuviéramos un solo tipo de canal, ahora qué pasa si consideramos que tenemos diferentes tipos de canales pasando iones, en condiciones distintas. 
Aquí es donde va a importarnos el hecho de que existen diferentes tipos de canales con voltajes de equilibrio diferente. 
Retomando a los potenciales de Nerst \(E_{Na},E_{K},E_{L}\) notemos que están dados por:

\begin{equation}
    E = \dfrac{k_{B}T}{zq}\ln\dfrac{[adentro]}{[afuera]}
 \label{eq:diferenciaP}
\end{equation}

Estos potenciales están relacionados con las características termodinámicas, en la ecuación anterior \ref{eq:diferenciaP} \(k_{B}\) la constante de Boltzman, \(q\) es la carga del ion, y \(z\) es el número de iones. El logaritmo natural representando el promedio de cuántos elementos tenemos en la parte de adentro con respecto a cuántos elementos tenemos en la parte de afuera.

Considerando los diferentes puntos de equilibrio en los cuales se puede encontrar la diferencia de potencial en la membrana, vamos a distinguir entre tres estados de esta (también se puede ver en \ref{fig:graficaP}):

\begin{enumerate}
 \item \textbf{Polarizada} en su estado de reposo con \(V < 0 ( V \approx -70mV )\).
 \begin{itemize}
  \item Su estado de reposo,cuando la neurona no está haciendo nada simplemente están corriendo los sodios y entran los potasios.
 \end{itemize}
 \item \textbf{Despolarizada} cuando \(V \geq 0\).
 \begin{itemize}
  \item Cuando en sus dendritas y en el cuerpo de la neurona se acumula una carga muy grande, se abre la compuerta de sodio y van a empezar a entrar un montón de sodio, esta diferencia de potencial que existía entre lo fuera y lo adentro se va a reducir de hecho se puede llegar a reducir bastante dependiendo de la carga que le estemos aplicando.
  \item Valores positivos en la diferencia de potencial.    
 \end{itemize}
 \item \textbf{Hiperpolarizada} cuando la diferencia de potencial incrementa su magnitud \(V << 0\).
 \begin{itemize}
 \item En cuanto se despolarice van a empiezar unas interacciones  entre los diferentes tipos de canales que lo que van a intentar hacer es regresar a la neurona en su estado normal.
 \item Si antes estaba quieta a los -70mV aquí, ahora va a quedar todavía más abajo alrededor de -90mV. Esto va a permitir un fenómeno que se le conoce como \emph{el periodo de refracción} y ese periodo sirve para que simplemente se lance un disparo y que el comportamiento eléctrico no se rebote otra vez en dirección contraria en la neurona, va a quedar muy quieta la neuronas durante un rato y después regresará otra vez es su equilibrio. 
 \end{itemize}

\end{enumerate}
 
\section{Modelo de las compuertas iónicas controladas por voltaje}

Retomando el modelo del circuito electrico modelando la membrana, junto con los canales y los iones, volvamos a verlo ahora en la \ref{fig:circuito1}

\begin{figure}[H]
 \centering
 \includegraphics[scale=0.5]{../Figuras/circuito.png}
 \caption{Modelo de la membrana axónica modelada como circuito eléctrico, con los distintos canales presentes y su voltaje de reposo.}
 \label{fig:circuito1}
\end{figure}

Recordemos brevemente las definiciones de los dos tipos de canales protagonistas en el modelo:

\begin{definition}
 \emph{Canal persistente} Tiene un sólo tipo de compuerta y dos estados posibles:
 \begin{enumerate}
  \item \textbf{Activado}
  \item \textbf{Desactivado}
 \end{enumerate}

\end{definition}

\begin{definition}
 \emph{Canal transitorio} Tiene compuertas de activación e inactivación, y tres estados:
 \begin{enumerate}
  \item \textbf{Activado} Ambas compuertas abiertas.
  \item \textbf{Desactivado} Compuerta de activación cerrada, inactivación abierta.
  \item \textbf{Inactivada} Compuerta de inactivación cerrada.
 \end{enumerate}

\end{definition}


Y retomando \hyperlink{LaEq}{la primera ecuación diferencial} donde tenemos por un lado la corriente que está pasando a través del capacitor y por otro lado vamos a tener las corrientes que están circulando a través de los diferentes canales, 

\begin{equation}
  C_{m} \dfrac{dV_{m}}{dt} =  - g_{Na} m^3 h(V_{m} - E_{Na} ) - g_{K} n 4 (V_{m} - E_{K} ) - g_{L} (V_{m} - E_{L} ) + I_ext
  \label{eq:corrientesRepaso}
\end{equation}

Las capacitancias y variables del lado izquierdo estan explicadas en la sección \hyperlink{secc}{\emph{Modelo de la membrana como bicapa de lípidos}}, aqui vamos a retomar las ecuaciones \ref{eq:corrientesEnLaMembrana3},\ref{eq:corrientesEnLaMembrana4},\ref{eq:corrientesEnLaMembrana5} de esa misma sección, (recordemos que estás ecuaciones describen la probabilidad de que los canales ionicos esten abiertos) que son las siguientes:
\begin{equation}
  \dfrac{1}{\gamma(T)}\dfrac{dn}{dt} =  \alpha_{n^\infty} (V)(1 - n) - \beta_{n} (V) n = \dfrac{n(V)-n(t)}{\tau_{n}(V)}
  \label{eq:probabilidades1}
\end{equation}

\begin{equation}
  \dfrac{1}{\gamma(T)}\dfrac{dm}{dt} =  \alpha_{m} (V)(1 - m) - \beta_{m} (V) m = \dfrac{m^\infty(V)-m(t)}{\tau_{m}(V)}
  \label{eq:probabilidades2}
\end{equation}

\begin{equation}
  \dfrac{1}{\gamma(T)}\dfrac{dh}{dt} =  \alpha_{h} (V)(1 - h) - \beta_{h} (V) h = \dfrac{h^\infty(V)-h(t)}{\tau_{h}(V)}
  \label{eq:probabilidades3}
\end{equation}

Ahora notemos los elementos en estás ecuaciones anteriores con \(ion\) pudiendo denotar las compuertas del potasio \(n\) o del sodio, ya sea \(m\) o \(h\):
\begin{itemize}
 \item \(\dfrac{1}{\gamma(T)}\) Este es el coeficiente de escala temporal, dependiente de la temperatura los por eso está apareciendo aquí una \(t\). Para las simulaciones que nosotros vamos a hacer vamos a pensar que estamos en una temperatura fija. 
 \item \(\alpha_{ion}(V)\) probabilidad de que una compuerta transite de abierta a cerrada.
 \item \(\beta_{ion}(V)\) probabilidad de que una compuerta transite de abierta a cerrada.
 \item \(ion^\infty(V)\) probabilidad de compuerta abierta en el equilibrio cuando \(t \rightarrow \infty\) .
 \item \((ion)\) Probabilidad de que cada compuerta (n,m,h) esté abierta.
 \item \((1-ion)\) Probabilidad de que cada compuerta (n,m,h) esté cerrada.
 
 \item \(\tau_{ion}(V)\) Tiempo que toma llegar al equilibrio.
\end{itemize}


Lo que vamos a ver es que forma de escribir la ecuación depende precisamente del número de compuertas que tenían para poder abrirse y cerrarse. Reescribir la ecuación de esta manera lo que nos permite es medirlo en términos de estas probabilidades de que se abran y cierren las compuertas que serían 

Esta probabilidad se empieza a alterar conforme cambiamos el voltaje pero no va a llegar a su valor de equilibrio sino hasta después de pasado un cierto periodo.


\section{Dinámica del voltaje durante un disparo} 
\begin{figure}[H]
 \centering
 \includegraphics[scale=0.5]{../Figuras/polarizacion1.png}
 \caption{Dinámica del voltaje.}
 \label{fig:voltaje1}
\end{figure}

\begin{figure}[H]
 \centering
 \includegraphics[scale=0.5]{../Figuras/polarizacion2.png}
 \caption{Pulso, cuando se rebasa el voltaje umbral, los canales de Na + y K + interactúan para producir
una rápida despolarización de la membrana, para luego hiperpolarizarla}
 \label{fig:voltaje1}
\end{figure}

\begin{figure}[H]
 \centering
 \includegraphics[scale=0.5]{../Figuras/medidasExperimentales.png}
 \caption{Medición experimental de los parámetros y ajuste manual de curvas. Imagen de Nelson
2004}
 \label{fig:voltajeAB}
\end{figure}

\begin{figure}[H]
 \centering
 \includegraphics[scale=0.5]{../Figuras/actinac.png}
 \caption{Las compuertas de Na + se abren primero, luego las de K + y esto provoca que se inactiven las de Na + .}
 \label{fig:voltajeActInac}
\end{figure}

\begin{figure}[H]
 \centering
 \includegraphics[scale=0.5]{../Figuras/disparo.png}
 \caption{Medidas experimentales de los estimulos y la reacción de los canales}
 \label{fig:voltajeActInac}
\end{figure}


\section{Simulación usando el método de Euler}
debemos alistarse rápidamente como estamos resolviendo esas ecuaciones diferenciales ya para tener una simulación numérica y se trata del algoritmo de integración de hoy no vamos a tener una solución explícita en nuestro sistema de ecuaciones sino que vamos a utilizar esta técnica en la que comenzamos con un valor inicial y a partir de ahí utilizamos las ecuaciones para calcular las tangentes aproximamos a la curva con su tangente y vamos avanzando paso a pasito entonces para ver cómo se comportan entonces estas gráficas que vamos a ver en la parte de arriba se obtuvieran precisamente con el método de boiler y ahorita le vamos a llamar la función integra disparo necesitamos cuatro valores que van a ir este que van a provocar diferentes comportamientos el primero es durante cuánto tiempo queremos correr la simulación el segundo que tan finos queremos que sean los pasos recordemos que vamos a aproximar la función con segmentos de recta siguiendo la tangente entonces si estos pasos son demasiado grandes y la curva ya se separó mucho del superagente entonces nuestra simulación va a ser mala y la función que obtengamos no se va a parece mucho la solución si hacemos pasos demasiado pequeños nos vamos a tardar demasiado en hacer el cómputo entonces hay que encontrar aquí un buen equilibrio entre ambos casos lo siguiente que vamos a necesitar es el voltaje inicial en donde empieza nuestra simulación donde estaba en nuestra neurona cuando empezamos a trabajar y el siguiente dato que necesitamos es la corriente externa de qué magnitud fue el toque que le estamos dando en este momento al action con estos datos entonces ya podemos ir calculando todos los demás elementos aquí ya no quise mencionar las constantes como las que me dieron joaquín y huxley precisamente porque son constantes entonces les pueden definir en cualquier lugar como parámetros fijos el siguiente paso bueno vamos a querer guardar lo que está ocurriendo para todos los tiempos desde cero hasta t en cada delta te lo estoy mencionando aquí como una serie de arreglos donde en la posición 0 viene nuestra primera medición en la posición 1 nuestra siguiente aproximación etcétera etcétera vamos a tener toda una serie de puntos donde estamos guardando estos pasos para inicializar oyler sabemos que necesitamos es un primer valor a partir del cual vamos a calcular la tangente y vamos a ir aproximando lo demás entonces para eso queríamos el voltaje inicial como nosotros sabemos donde estaba en reposo nuestra célula originalmente vamos a poder guardar ese voltaje como el primer valor para nuestra simulación ahora todos los demás elementos las alfa las betas etcétera etcétera se pueden calcular si ya conocíamos ese voltaje inicial entonces a partir de este momento podemos repetir el mismo ciclo tantas veces como sea necesario para cubrir el intervalo desde el tiempo inicial hasta el tiempo t brincando del rate en del tati entonces dado un voltaje vamos a calcular las diferentes alfaz que son las que se medían experimentalmente originalmente utilizando las ecuaciones que encontraron joaquín y jujuy a partir de estas el fasi estas vetas entonces ahora si podemos calcular las nm h todas las dos que son las que estamos usando acá ya teniendo estas entonces podemos calcular las probabilidades para las compuertas nm h utilizando las ecuaciones en diferencias en forma matricial ahora aquí esta parte fue importante sobre todo por el asunto de las pérdidas numéricas es recordemos que la computadora tiene una representación en punto flotante lo cual quiere decir que un número real no se puede representar en la computadora voy a pasar a la siguiente cuando lleguemos al cálculo de las generadas las secas y lleguemos finalmente a calcular el voltaje b nos va a importar exactamente como escribimos los términos por el problema del truncamiento a qué me refiero supongamos que tenemos un número real este número no lo podemos representar en la computadora en algún momento se nos acaba el espacio y vamos a tener que volar nos todos los dígitos que hubieran seguido acá concretamente se va a notar eso mucho aquí en la simulación del modelo de hodgkin y hawkes y entonces cada vez que nosotros tocamos estos dígitos estamos perdiendo precisión si nosotros por ejemplo en vez de dividir 3.14 16 entre 2.25 hacemos esto 35 en una computadora vamos a obtener este resultados diferentes por eso que les dimos si les estamos diciendo cuando separar cuando no separar para que todos podamos obtener los mismos resultados desde más ojo con esa parte porque a veces puede ser que no les salgan los disparos pero porque primero dicen ha de sumar lo que será la división o primero la división y luego era de suma y estaba había que calculando en el orden diferente y es por eso les estamos poniendo aquí paso por paso cómo le tenemos que hacer en qué orden para hacer los cálculos es bueno una vez que ya terminamos de calcular estos términos que se van a necesitar en la ecuación más grande que es la del voltaje de la membrana podemos ir almacenando en los resultados temporales dentro de nuestros arreglos en la casilla que les corresponda para ese paso y aquí es donde ya vamos a utilizar la corriente externa para meterla en la ecuación diferencial para el voltaje una vez que tengamos esto tenemos que repetirlo para cada paso los observen que esto es como que para el tiempo t se calcula entonces estas y esa onda va a dar el siguiente valor del voltaje ya teniendo el siguiente valor del voltaje lo voy a poder utilizar para el siguiente paso y así nos vamos a seguir todo el tiempo si devolvemos estos bueno después ya nos vamos a poder permitir graficar qué fue lo que sucedió con cada uno de ellos y eso es precisamente de dónde sale esta imagen que tenemos aquí y todo lo referente a la simulación numérica.

\section{Información condificada en las dendritas}

