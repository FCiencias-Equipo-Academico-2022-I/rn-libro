\section{Compuertas lógicas con neuronas}

McCulloch y Pitts demostraron en 1943 que mediante  el perceptrón se podía realizar cualquier función lógica.
Aquí se muestra como se puede utilizar un perceptrón para simular compuertas lógicas tales como el and y el or.

Tomando el hecho que en la naturaleza las neuronas van pasando información en una estructura que forma niveles de abstración, esto lo modelamos como capas de neuronas conectadas entre sí, cada capa haciendo su trabajo de abstración.
El perceptrón simple es un modelo neuronal unidireccional, de dos capas de neuronas, una de entrada y otra de salida. La operación de una red de este tipo, con n neuronas de entrada y m neuronas de salida, se puede expresar de la siguiente forma:
