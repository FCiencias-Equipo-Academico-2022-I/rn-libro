\chapter{Factores}

\begin{definition}[Factor]
Un factor $\phi(X_1,...,X_k)$, sobre variables $X_i$, donde cada $X_i$ puede tomar valores de un dominio $D_{X_i}$ es una función:

\begin{align}
 \phi : Val(X_1,...,X_k) \rightarrow \mathbb{R}
\end{align}
que asocia un número real a posibles asignaciones de valores a las variables $X_1,...,X_k$.  El \emph{alcance} $\mathbb{A}$ de un factor son las variables cuyos posibles valores están siendo considerados.
\begin{align}
 \mathbb{A}(\phi(X_1,...,X_k)) = \{X_1,...,X_k\}
\end{align}
\end{definition}


\begin{example}
Un factor.  Sean las variables y dominios:
\begin{itemize}
 \item $Estaci\acute{o}n$, $D_{Estaci\acute{o}n}=\{Primavera, Verano, Oto\tilde{n}o,Invierno\}$
 
 \item $Lluvia$, $D_{Lluvia}=\{0,1\}$
\end{itemize}
de modo que el alcance es $\mathbb{A} = \{Estaci\acute{o}n, Lluvia\}$

\begin{center}
\rowcolors{2}{\colTableRow}{white}
\begin{tabular}{cc|c}
 $Estaci\acute{o}n$ & $Lluvia$ & Frecuencia (días/mes) \\ \hline
 Primavera & 0 & 18 \\
 Primavera & 1 & 6 \\
 Verano & 0 & 7 \\
 Verano & 1 & 17 \\
 Otoño & 0 & 17 \\
 Otoño & 1 & 7 \\
 Invierno & 0 & 20 \\
 Invierno & 1 & 5 \\
\end{tabular}
\end{center}
\end{example}


Se utilizarán factores para representar las distribuciones de probabilidad, distribuciones de probabilidad conjuntas y condicionales.

\section{Marginalización}

\begin{definition}[Marginalización]
Dada una variable $X_i$, en el alcance $\mathbb{A}$ de un factor $\phi = \phi(X_1,...,X_k)$, que se desea marginalizar, se define a la operación como:
\begin{align}
 marginalizaci\acute{o}n(\phi(X_1,...,X_k), X_i) =& \phi'(\mathbb{A'}) \\
 \mathbb{A'} =& \mathbb{A}(\phi) - X_i \text{ con } i \in [1,k] \\
 \phi'(x_1,...,x_{i-1},x_{i+1},...,x_k) =& \sum_{Val\{X_i\}} \phi(x_1,...,x_k)
\end{align}
donde $x_i$ es un valor particular de $X_i$, $\phi'(x_1,...,x_{i-1},x_{i+1},...,x_k)$ es el renglón del factor $\phi'$ con $\{X_1 = x_i,..., X_k=x_k\}$ y $Val\{X_i\}$ es el conjunto de valores posibles asignables a $X_i$.
\end{definition}

\begin{example}{Marginalizar $Estaci\acute{o}n$}
\begin{center}
\rowcolors{2}{\colTableRow}{white}
\begin{tabular}{cc|c}
 $Estaci\acute{o}n$ & $Lluvia$ & $P(Lluvia,Estaci\acute{o}n)$ \\ \toprule
 Primavera & 0 & 0.1875 \\
 Primavera & 1 & 0.0625 \\
 Verano & 0 & 0.075 \\
 Verano & 1 & 0.175 \\
 Otoño & 0 & 0.175 \\
 Otoño & 1 & 0.075 \\
 Invierno & 0 & 0.2 \\
 Invierno & 1 & 0.05 \\
 \multicolumn{2}{c}{}  & $\sum=1$
\end{tabular}$\Rightarrow$\rowcolors{2}{\colTableRow}{white}
\begin{tabular}{c|c}
 $Lluvia$ & $P(Lluvia)$ \\ \toprule
 0 & 0.63750 \\
 1 & 0.36250 \\
 \multicolumn{1}{c}{}  & $\sum=1$
\end{tabular}
\end{center}
\end{example}


\section{Reducción}

\begin{definition}[Reducción]
Dado un valor $x_i = a$ para una de las variables $X_i$ en el alcance $\mathbb{A}$ del factor $\phi$, se reduce el factor eliminando todas aquellas entradas en las cuales no se cumple que $X_i = a$.  La operación se define como:
\begin{align}
 reducci\acute{o}n(\phi(X_1,...,X_k), X_i, a) =& \phi'(\mathbb{A'}) \\
 \mathbb{A'} =& \mathbb{A}(\phi) - X_i \text{ con } i \in [1,k] \\
 \phi'(x_1,...,x_{i-1},x_{i+1},...,x_k) =& \phi(x_1,...,x_k) \text{ con } X_i = a
\end{align}
\end{definition}

A diferencia de las distribuciones de probabilidad, la reducción en un factor no requiere renormalizar sus valores asociados, pues por definición, no es necesario que éstos sumen uno.

\begin{example}{Reducir $Estaci\acute{o}n = Primavera$}
\begin{center}
\begin{tabular}{cc|c}
 $Estaci\acute{o}n$ & $Lluvia$ & $P(Lluvia,Estaci\acute{o}n)$ \\ \toprule
 \rowcolor{\colTableRow} Primavera & 0 & 0.1875 \\
 \rowcolor{\colTableRow} Primavera & 1 & 0.0625 \\
 Verano & 0 & 0.075 \\
 Verano & 1 & 0.175 \\
 Otoño & 0 & 0.175 \\
 Otoño & 1 & 0.075 \\
 Invierno & 0 & 0.2 \\
 Invierno & 1 & 0.05 \\
 \multicolumn{2}{c}{}  & $\sum=1$
\end{tabular}$\Rightarrow$
\begin{tabular}{c|c}
 $Lluvia$ & $P(Lluvia,Estaci\acute{o}n = primavera)$ \\ \toprule
 0 & 0.1875 \\
 1 & 0.0625 \\
 \multicolumn{1}{c}{}  & $\sum=0.25$
\end{tabular}
\end{center}
\end{example}


\section{Normalización}

\begin{definition}[Normalización]
Dado un factor $\phi$ con $n$ renglones sea
\begin{align}
 s =& \sum_{i=1}^n \phi_i
\end{align}
con $\phi_i$ el valor asociado al renglón $i$, $s$ es la suma de los valores de todos los renglones.  Entonces:
\begin{align}
 normalizaci\acute{o}n(\phi) =& \phi' \\
 \phi'_i = \dfrac{\phi_i}{s}
\end{align}
donde el valor en cada renglón de $\phi$ ha sido dividido entre $s$.
\end{definition}

\begin{example}{Normalizar}
\begin{center}
\begin{tabular}{c|c}
 $Lluvia$ & $P(Lluvia,primavera)$ \\ \toprule
 0 & 0.1875 / 0.25 \\
 1 & 0.0625 / 0.25 \\
 \multicolumn{1}{c}{}  & $\sum=0.25$
\end{tabular}$\Rightarrow$
\begin{tabular}{c|c}
 $Lluvia$ & $P(Lluvia|primavera)$ \\ \toprule
 0 & 0.75 \\
 1 & 0.25 \\
 \multicolumn{1}{c}{}  & $\sum=1$
\end{tabular}
\end{center}
\end{example}


\section{Multiplicación}

\begin{definition}[Multiplicación]
Se define el producto de factores de tal modo que:
\begin{align}
 \phi_1 =& \phi_1(X_1,...,X_k) \\
 \phi_2 =& \phi_2(Y_1,...,Y_l) \\
 \phi =& \phi_1 \phi_2 \\
 \mathbb{A}(\phi) =& \mathbb{A}(\phi_1) \cup \mathbb{A}(\phi_2) \\
 \phi(z_1,...,z_m) =& \phi_1(x_1,...,x_k) * \phi_2(y_1,...,y_l)
\end{align}
donde $Z_k \in \mathbb{A}(\phi)$ por lo que:
\begin{align}
 \text{ si } Z_k \in \mathbb{A}(\phi_1) & \Rightarrow Z_k = X_i \land z_k = x_i \\
 \text{ si } Z_k \in \mathbb{A}(\phi_2) & \Rightarrow Z_k = Y_j \land z_k = y_j
\end{align}
obsérvese que si $Z_k \in \mathbb{A}(\phi_1)$ y $Z_k \in \mathbb{A}(\phi_2)$ entonces $Z_k = X_i = Y_j$ y en cada renglón deberá cumplirse $z_k = x_i = y_j$.
\end{definition}


\begin{example}{Multiplicar $P(Lluvia|Estaci\acute{o}n)P(Estaci\acute{o}n)$}
\begin{center}
\begin{tabular}{cc|c}
 $Estaci\acute{o}n$ & $Lluvia$ & $P(Lluvia|Estaci\acute{o}n)$ \\ \toprule
 \rowcolor{\colTableRow}Primavera & 0 & 0.75 \\
 \rowcolor{\colTableRow}Primavera & 1 & 0.25 \\
 \rowcolor{\colTableRowOne}Verano & 0 & 0.30 \\
 \rowcolor{\colTableRowOne}Verano & 1 & 0.70 \\
 \rowcolor{\colTableRowTwo}Otoño & 0 & 0.70 \\
 \rowcolor{\colTableRowTwo}Otoño & 1 & 0.30 \\
 \rowcolor{\colTableRowThree}Invierno & 0 & 0.80 \\
 \rowcolor{\colTableRowThree}Invierno & 1 & 0.20
\end{tabular}$\times$
\begin{tabular}{c|c}
 $Estaci\acute{o}n$ & $P(Estaci\acute{o}n)$ \\ \toprule
 \rowcolor{\colTableRow}Primavera & 0.25 \\
 \rowcolor{\colTableRowOne}Verano & 0.25 \\
 \rowcolor{\colTableRowTwo}Otoño & 0.25 \\
 \rowcolor{\colTableRowThree}Invierno & 0.25 \\
 \multicolumn{1}{c}{}  & $\sum=1$
\end{tabular}$=$

\begin{center}
\begin{tabular}{cc|c}
 $Estaci\acute{o}n$ & $Lluvia$ & $P(Lluvia,Estaci\acute{o}n)$ \\ \toprule
 \rowcolor{\colTableRow}Primavera & 0 & 0.1875 \\
 \rowcolor{\colTableRow}Primavera & 1 & 0.0625 \\
 \rowcolor{\colTableRowOne}Verano & 0 & 0.075 \\
 \rowcolor{\colTableRowOne}Verano & 1 & 0.175 \\
 \rowcolor{\colTableRowTwo}Otoño & 0 & 0.175 \\
 \rowcolor{\colTableRowTwo}Otoño & 1 & 0.075 \\
 \rowcolor{\colTableRowThree}Invierno & 0 & 0.2 \\
 \rowcolor{\colTableRowThree}Invierno & 1 & 0.05 \\
 \multicolumn{2}{c}{}  & $\sum=1$
\end{tabular}
\end{center}
\end{center}
\end{example}


\begin{example}{Multiplicar $P(Lluvia)P(Estaci\acute{o}n)$}
\begin{center}
\begin{tabular}{c|c}
 $Lluvia$ & $P(Lluvia)$ \\ \toprule
 \rowcolor{gray!10}0 & 0.63750 \\
 1 & 0.36250 \\
 \multicolumn{1}{c}{}  & $\sum=1$
\end{tabular}
$\times$
\begin{tabular}{c|c}
 $Estaci\acute{o}n$ & $P(Estaci\acute{o}n)$ \\ \toprule
 \rowcolor{\colTableRow}Primavera & 0.25 \\
 \rowcolor{\colTableRowOne}Verano & 0.25 \\
 \rowcolor{\colTableRowTwo}Otoño & 0.25 \\
 \rowcolor{\colTableRowThree}Invierno & 0.25 \\
 \multicolumn{1}{c}{}  & $\sum=1$
\end{tabular}$=$

Sea $R = P(Lluvia)P(Estaci\acute{o}n)$

\begin{center}
\begin{tabular}{cc|c}
 $Estaci\acute{o}n$ & $Lluvia$ & $R$ \\ \toprule
 \rowcolor{\colTableRow}Primavera & 0 & 0.159375 \\
 \rowcolor{\colTableRow!30}Primavera & 1 & 0.090625 \\
 \rowcolor{\colTableRowOne}Verano & 0 & 0.159375 \\
 \rowcolor{\colTableRowOne!30}Verano & 1 & 0.090625 \\
 \rowcolor{\colTableRowTwo}Otoño & 0 & 0.159375 \\
 \rowcolor{\colTableRowTwo!30}Otoño & 1 & 0.090625 \\
 \rowcolor{\colTableRowThree}Invierno & 0 & 0.159375 \\
 \rowcolor{\colTableRowThree!30}Invierno & 1 & 0.090625 \\
 \multicolumn{2}{c}{}  & $\sum=1$
\end{tabular}$=$
\begin{tabular}{cc|c}
 $Estaci\acute{o}n$ & $Lluvia$ & $R$ \\ \toprule
 \rowcolor{\colTableRow}Primavera & 0 & 0.159375 \\
 \rowcolor{\colTableRowOne}Verano & 0 & 0.159375 \\
 \rowcolor{\colTableRowTwo}Otoño & 0 & 0.159375 \\
 \rowcolor{\colTableRowThree}Invierno & 0 & 0.159375 \\
 \rowcolor{\colTableRow!30}Primavera & 1 & 0.090625 \\
 \rowcolor{\colTableRowOne!30}Verano & 1 & 0.090625 \\
 \rowcolor{\colTableRowTwo!30}Otoño & 1 & 0.090625 \\
 \rowcolor{\colTableRowThree!30}Invierno & 1 & 0.090625 \\
 \multicolumn{2}{c}{}  & $\sum=1$
\end{tabular}
\end{center}
\end{center}
\end{example}


\section{Distribuciones de probabilidad con factores}

Claramente, los factores pueden contener, en particular, tablas de distribuciones de probabilidad sobre variables discretas.  Las variables aleatorias del sistema aparecerán en el alcance del factor, las combinaciones posibles de asignaciones a estas variables quedarán registradas en los renglones del factor y la probabilidad de que ocurran será el número real asociado a esa entrada.


\subsection{Distribuciones de probabilidad condicional con factores}

Para escribir la distribución de probabilidad condicional utilizando factores, la notación cambia un poco.  Por ejemplo:

\begin{center}
\begin{tabular}{cc|cc}
 $Estaci\acute{o}n$ & $Lluvia$ & $P(Lluvia|Estaci\acute{o}n)$ \\ \toprule
 Primavera & 0 & 0.75 & \multirow{2}{*}{$\sum=1$} \\
 Primavera & 1 & 0.25 \\ \arrayrulecolor{gray}\cline{1-3}
 Verano & 0 & 0.30 & \multirow{2}{*}{$\sum=1$} \\
 Verano & 1 & 0.60 \\ \arrayrulecolor{gray}\cline{1-3}
 Otoño & 0 & 0.70 & \multirow{2}{*}{$\sum=1$} \\
 Otoño & 1 & 0.30 \\ \arrayrulecolor{gray}\cline{1-3}
 Invierno & 0 & 0.8 & \multirow{2}{*}{$\sum=1$} \\
 Invierno & 1 & 0.1 \\
\end{tabular}
\end{center}
