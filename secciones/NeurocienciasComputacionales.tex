\section{Neurociencias computacionales}

Las redes neuronales surgieron completamente inspiradas en los sistemas biológicos. 
Lo que estamos haciendo los computólogos es tomar una idea a la naturaleza, una idea que ha probado ser sumamente efectiva para procesar información y que logra resolver problemas que nosotros aún no sabemos hacer con modelos diseñados explícitamente.
Los más notorios: 
\begin{itemize}
\item Problemas de visión por computadora 
\item Procesamiento del lenguaje natural 
\end{itemize}

A lo largo del texto tendremos una somera idea de que hace el sistema nervioso de un ser humano, tomaremos también ejemplos de animales como, el calamar gigante, cangrejos. Ejemplos que han permitido estudiar biológicamente, cómo funcionan las neuronas y cómo funciona su sistema nervioso. 

Entonces por un momento veamos el sistema como un todo, lo que realmente está haciendo en el cómputo no es de una sola neurona sino de la colección de todos ellas, lo que sucede con los sistemas biológicos es que son muchísimo más complicados que lo que vamos a ver nosotros como modelos computacionales sin embargo muchísimas empresas están utilizando estas técnicas. 
El sistema nervioso como un todo es bastante más complejo pero conforme han ido evolucionando las redes neuronales computacionales ya con sus arquitecturas y organizaciones, se están volviendo también más complejas.Varias de las estructuras más exitosas tienen un análogo muy fuerte con un sistema nervioso natural. 

Veamos un campo conocido como neurociencias computacionales el cual se dedica explícitamente al estudio/modelo de los sistemas biológicos pero ya conjuntando varios campos. Se van a interesar notablemente en:  descripciones y modelos funcionales biológicamente realistas de neuronas y sistemas neuronales. Lo que veremos en redes neuronales computacionales no necesariamente tienen que ser realistas, lo que nos interesa es que resuelvan los problemas, si se desvían un poco de cómo funcionan los sistemas naturales en un principio no es problema. 

Ahora, ¿Qué les interesa modelar? Se fijan en la fisiología y en la dinámica de estos sistemas, combinan varias ciencias tales como: 
\begin{itemize}
 \item  \textbf{Biofísica} 
 \item  \textbf{Neurociencias tradicionales} con modelos matemáticos. 
 \item  \textbf{Ciencias de la computación} tanto en la parte del modelado como en la parte de la implementación de estos modelos y la generación de simulaciones computacionales.
 \item  \textbf{Ingeniería eléctrica} se está diseñando hardware especializado para ejecutar modelos de manera eficiente, algunos de los modelos matemáticos están basados en circuitos eléctricos. 
 \item  \textbf{Ciencias cognitivas} tratan de ver que se está codificando dentro de un sistema nervioso y cómo podemos interpretar esa información que está ahí guardada.

\end{itemize}

De entre todo esto vamos a ver cómo está influyendo todo esto, en lo que va a hacer las ciencias de la computación pero con su propio modelo de redes neuronales (Existe una conexión muy fuerte entre estos dos campos).


Las neurociencias computacionales como se mencionó anteriormente estudia modelos del sistema nervioso y clasifica estos modelos en tres tipos: 
\begin{itemize}
 \item \textbf{Modelos descriptivos}, nos limitamos a decir que está haciendo un sistema y en particular aquí son muy famosos los experimentos con ratones se está tratando de ver qué puede hacer, que no puede hacer, que puede aprender, que no pero no no se puede explicar “como”, simplemente se dice que es lo que está sucediendo. 
 \item \textbf{Modelos mexicanistas}, donde ahora sí nos interesa saber cómo es que están haciendo las cosas aquí vamos a ver los modelos matemáticos que precisamente nos están tratando de describir cómo puede ser que se están conectando estas neuronas, cómo pueden estar funcionando las redes de neuronas, cómo podría estarse almacenando la información y transfiriendo de un lado a otro. 
 \item \textbf{Modelos interpretativos}, nos dan una idea del por qué o para qué lo hacen. Se tiene que buscar intencionalidad, razonamiento de más alto nivel.
\end{itemize}

Cuando trabajemos con en redes de computadoras vamos a notar que sí necesitamos trabajar un poco con los niveles 2 y 3. Para romper ese mito que nuestras redes neuronales, sabemos que aprendieron y no estamos ni siquiera seguros de que aprendieron o porque lo aprendieron así, vamos a tener que utilizar herramientas matemáticas para tratar de descubrir qué es lo que realmente está haciendo la red entrenada. 


Ahora los \textbf{objetivos del modelado}:

\textcolor{gray}{(Empezando desde lo más granular que es cada una de las neuronas)}

\begin{itemize}
 \item Las \textbf{corrientes}, que están pasando a través de las membranas de las neuronas, la influencia que tiene en el paso de la información. 
 \item Las \textbf{proteínas buenas} van a jugar un papel importante en la conducción de elementos iónicos no transmisores (acoplamientos químicos).
\end{itemize}

\textcolor{gray}{(El siguiente nivel ya no solamente de una neurona)}

\begin{itemize}
 \item Las \textbf{oscilaciones de las redes} completas, que pasa con estas señales, pulsos eléctricos, que se están transfiriendo de unas regiones a otras y que empiezan a producir oscilaciones con ciertos períodos,  regiones de actividad, que se apagan.
 \item \textbf{Arquitectura topográfica y de columnas} cómo están organizadas estas neuronas, quienes están conectadas con quiénes, cómo reaccionan dentro de ciertas regiones identificadas, cómo interactúan con otras regiones. Se puede identificar una arquitectura desde el punto de vista fisiológico como de vista funcional. Un caso particular de estas estructuras es la formación de columnas de neuronas que están altamente conectadas y trabajan como una unidad.
 \item El \textbf{aprendizaje} es decir estamos procesando información, estamos guardando información, recuperando y eso permite que los seres que cuentan con un sistema nervioso tengan características especiales cuyo comportamiento se puede modificar conforme aprenden. 
 \item La \textbf{memoria} que significa que necesitamos almacenar información, recuperarla procesarla. 
\end{itemize}



