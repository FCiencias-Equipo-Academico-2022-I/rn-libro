\documentclass{article}
\usepackage[spanish, mexico]{babel}
\usepackage{graphicx} % Required for inserting images
\usepackage{biblatex} %Imports biblatex package
\addbibresource{NC_01.bib} %Import the bibliography file

\title{UNO}
\author{fernanda.j.gtz.k }
\date{October 2023}

\begin{document}

\maketitle

%\section{Introduction}



\section{Neurociencias computacionales}

%Referencia a libro uno \cite{libroUNO}


Redes neuronales artificiales, inspiradas de las redes neuronales biológicas, tales como las de los animales %\cite{incipiencias}
, en un primer momento, hasta llegar al ser humano, con nuestras neuronas %\cite{ideaNeurona}. 

Nuestro objetivo principal en el curso es dar soluciones, tomando la idea de la naturaleza. Si el cerebro humano funciona, la imitación debe darnos algunos resultados, idea que ha funcionado para procesar información y dar solución con modelos diseñados. 

Los problemas más notorios a resolver son: 
\begin{itemize}
\item Problemas de visión por computadora.
\item Procesamiento del lenguaje natural.
\end{itemize}

La solución acordada es enfocarnos en todo el sistema nervioso, son muchas partes que nos conforman y nos dan vida. Si nos enfocamos en el cerebro, una parte es encargada de ver, otra de oír, y de alguna forma lo hacemos al mismo tiempo (3). Tal como una red, al jalar, ¿cómo atrapó al pesado? ¿Una cuerda? Muchas cuerdas conectadas. Entonces lo que realmente está pasando al computar una red neuronal artificial(4), es cada neurona tiene una tarea simple, que pasa a otra neurona, hasta que en conjunto se logró una tarea compleja. 

Imitar enteramente el sistema nervioso es una tarea ardua, pero posible, pero empecemos por la neurona.

El campo conocido como \textbf{neurociencias computacionales}(5) es que se dedica explícitamente al estudio/modelado de los sistemas biológicos con ayuda de varios campos de estudio. Se interesa notablemente en descripciones y modelos funcionales biológicamente realistas de neuronas y sistemas neuronales. Los modelos de redes neuronales que tomaremos, son burdos. Distan mucho de los sistemas naturales y aun así dan solución. Lo que nos interesa es que resuelvan los problemas inmediatos y de corto plazo.

Los campos de estudio de los cuales nos ayudamos son:
\begin{itemize}
 \item  \textbf{Ciencias cognitivas} dedicadas a tratar directamente con los humanos, de estás tomamos los conceptos de aprendizaje.
 \item  \textbf{Biofísica}  por el estudio de las propiedades físicas en de los sistemas biológicos. 
 \item  \textbf{Neurociencias tradicionales} con modelos matemáticos. 
 
 \item  \textbf{Ciencias de la computación} modelar e implementar los modelos dados, por las aréas aquí listadas. Simulaciones computacionales. 
 
 \item  \textbf{Ingeniería eléctrica} diseño de hardware especifico y eficiente para tomar los pulsos y medidas exactas.  
 
\end{itemize}

Las neurociencias computacionales, estudian modelos del sistema nervioso y clasifican estos modelos en tres tipos, recordemos claro el método cientifico: 

\begin{enumerate}
 \item \textbf{Modelos descriptivos},Describe. Por ejemplo describe el comportamiento de los ratones a ciertas sustancias. La comparación, que estaba pasando antes de la sustancia, con la sustancia y despues. 
  
 \item \textbf{Modelos mecanistas}, Mecánicamente %\cite{modeloM}
 , este lo ve a partir de  ¿cómo es que están haciendo las cosas? Siguiendo con el ejemplo. El ratón se cayo y se levanto, ¿Cómo? Doblo sus articulaciones, tomo impulso con sus musculos, roto su cuerpo, hasta reicorporarse totalmente.
 
 \item \textbf{Modelos interpretativos}, Se interpreta. ¿Porque hizo los movimientos mecanicos?, El ratón quiere regresar a su estado anterior antes de la pertuvación por la sustancia. ¿Sera verdad? Es lo que se interpreta, no la verdad absoluta. Se tiene que buscar intencionalidad, razonamiento de más alto nivel.
\end{enumerate}

Los \textbf{objetivos del modelado} en neurociencias:

%\textcolor{gray}{(Empezando desde lo más granular que es cada una de las neuronas)}

\begin{itemize}
 \item Las \textbf{corrientes} que están pasando a través de las membranas de las neuronas, la influencia que tienen en el paso de la información. 
 
 \item Las \textbf{proteínas} van a jugar un papel importante en la conducción de elementos iónicos, neurotransmisores y en los acoplamientos químicos.
\end{itemize}

%\textcolor{gray}{(El siguiente nivel con combinaciones de varias neuronas)}

\begin{itemize}
 \item Las \textbf{oscilaciones de las redes} completas, qué pasa con estas señales, pulsos eléctricos que se están transfiriendo de unas regiones a otras y que empiezan a producir oscilaciones con ciertos períodos,  regiones de actividad o regiones que se apagan.
 
 \item \textbf{Arquitectura topográfica y de columnas} cómo están organizadas estas neuronas, quiénes están conectadas con quiénes, cómo reaccionan dentro de ciertas regiones identificadas, cómo interactúan con otras regiones. Se puede identificar una arquitectura tanto desde el punto de vista fisiológico como del punto de vista funcional. Un caso particular de estas estructuras es la formación de columnas de neuronas que están altamente conectadas y trabajan como una unidad.
 
 \item El \textbf{aprendizaje}, es decir estamos procesando información, guardando información, recuperándola y eso permite que los seres que cuentan con un sistema nervioso tengan características especiales cuyo comportamiento se puede modificar conforme aprenden. 
 
 \item La \textbf{memoria}, necesitamos almacenar información y recuperarla para procesarla. 
\end{itemize}


\printbibliography[keyword={latex},title={Referencias}]
\end{document}

