\section{Neurociencias computacionales}

El campo conocido como \textbf{neurociencias computacionales}\cite{sonNC} es el que se dedica explícitamente al estudio/modelado de los sistemas biológicos con ayuda de varios campos de estudio. Se interesa notablemente en descripciones y modelos funcionales biológicamente realistas de neuronas y sistemas neuronales. 

Los campos de estudio de los cuales nos ayudamos son:
\begin{itemize}
 \item  \textbf{Ciencias cognitivas} dedicadas a tratar directamente con los humanos, de estás tomamos los conceptos de aprendizaje.
 \item  \textbf{Biofísica}  por el estudio de las propiedades físicas en de los sistemas biológicos. 
 \item  \textbf{Neurociencias tradicionales} con modelos matemáticos. 
 
 \item  \textbf{Ciencias de la computación} modelar e implementar los modelos dados, por las aréas aquí listadas. Simulaciones computacionales. 
 
 \item  \textbf{Ingeniería eléctrica} diseño de hardware especifico y eficiente para tomar los pulsos y medidas exactas.  
 
\end{itemize}

Las neurociencias computacionales, estudian modelos del sistema nervioso y clasifican estos modelos en tres tipos, recordemos el método cientifico: 

\begin{enumerate}
 \item \textbf{Modelos descriptivos}, Describen. Por ejemplo describe el comportamiento de los ratones a ciertas sustancias. La \textbf{comparación}, que estaba pasando antes de la sustancia, con la sustancia y después. 
  
 \item \textbf{Modelos mecanistas}, Mecánicamente. El \textbf{cómo} paso el evento o la acción. Siguiendo con el ejemplo. El ratón se cayo y se levanto, ¿Cómo? Doblo sus articulaciones, tomo impulso con sus musculos, roto su cuerpo, hasta reicorporarse totalmente.
 
 \item \textbf{Modelos interpretativos}, Se interpreta. ¿\textbf{Porque} hizo los movimientos mecanicos?. Siguiendo con el ejemplo. ¿Porque el ratón se reincorporo? Porque el ratón quiere regresar a su estado anterior antes de la pertubación por la sustancia. ¿Sera verdad? Es lo que se interpreta, no la verdad absoluta. Se tiene que buscar intencionalidad, razonamiento de más alto nivel.
 
\end{enumerate}

Los \textbf{objetivos del modelado} en neurociencias de nuestro interes en el curso son; Las corrientes, las proteínas, las oscilaciones de las redes completa, la arquitectura topografica y de columnas, el aprendizaje y la memoria. Por lo menos para el curso de Redes Neuronales en la Universidad Nacional Autonoma de México con la profesora Veronica Arriola Rios.

Para más información en Neurociencias \cite{princNS5}, siempre está el interes propio.

Las redes neuronales artificiales (RNA), estan inspiradas en las redes neuronales biológicas, tales como las de los animales \cite{incipiencias}, en un primer momento. Hasta llegar al ser humano con el sistema nervioso y nuestras neuronas. \cite{ideaNeurona}. 

Los modelos de redes neuronales que tomaremos, son simples. Distan mucho de los sistemas naturales y aun así dan solución. Lo que nos interesa es que resuelvan los problemas inmediatos y de corto plazo. Si el cerebro humano funciona, la imitación debe darnos algunos resultados, idea que ha funcionado para procesar información y dar solución con modelos diseñados. 

Los problemas más notorios a resolver son: 
\begin{itemize}
\item Problemas de visión por computadora.
\item Procesamiento del lenguaje natural.
\end{itemize}
