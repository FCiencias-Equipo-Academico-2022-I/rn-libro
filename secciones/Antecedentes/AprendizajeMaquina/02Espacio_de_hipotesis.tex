\section{Espacio de Hipotesis}

El aprendizaje automático, es utilizar datos disponibles para, aprender una tarea mediante una función que mejor mapee entradas a ciertas salidas. A esto se le llama  aproximación de función, en el que aproximamos una función de destino desconocida (que suponemos que existe) que puede asignar mejor las entradas a las salidas en todas las observaciones posibles del dominio del problema.

Una función de un modelo que se aproxima a la función objetivo y realiza asignaciones de entradas a salidas se denomina hipótesis.

Ahora estas funciones pueden tener formas muy generales en el aprendizaje de máquina pueden tener forma, por ejemplo, de estructuras de datos, como los árboles de decisión, donde cada nodo pregunta si o no, pertenece una clasificación.
pueden ser también funciones matemáticas como el caso de las redes neuronales, entonces la forma que tomen estas hipótesis en general puede abarcar muchos métodos y estructuras. 

Entonces el aprendizaje consiste en, explorar un espacio de posibles hipostesis para encontrar la hipotesis (una función) que mejor encaje, deacuerdo a lo se obtuvo en los ejemplos de entrenamiento, y predecir alguna característica de salida deseada. Usualmente se denotan como sigue:

\begin{itemize}
 \item \emph{h} (hipótesis): una sola hipótesis, por ejemplo una instancia o modelo candidato específico que asigna entradas a salidas, se puede evaluar y se usa para hacer predicciones.

 \item \emph{H} ( conjunto de hipótesis ): Un espacio de posibles hipótesis para mapear entradas.
 \end{itemize}


Una breve ejemplo para denotar un espacio de hipostesis sería el problema es saber los días que nos conviene ir al cine,
donde nuestra tarea T es aprender a predecir el conjunto de dias que nos conviene ir al cine, basado en los atributos de los dias, donde cada hipotesis la representaremos apartir de un conjunto de atributos de las instancias (dias), estonces cada hipotesis es un vector con tres atributos, \emph{tiene2x1, esEstrenoDePelicula, actoresConocidos}. Para cada atributo de la hipotesis tendría uno de los siguientes valores; $Si, No, ?$. Donde ? indica que cualquier valor es valido para ese atributo.

Cuando alguna instancia $x$ cumpla con todos los atributos de una \(h\), entonces \(h(x) = 1 \) y $x$ es un ejemplo positivo. Entonces para representar la hipostesis, que nos conviene ir solo los dias con 2x1, y que hay peliculas donde los actores son conocidos, la escribimos como $h(\textlangle Si, ?, Si\textrangle) = 1 $, la hipotesis que cualquier día nos conviene ir al cine la denotamos como $h(\textlangle ?, ?, ?\textrangle) = 1 $, nuestra función objetivo la denotamos como una función booleana $c:X \rightarrow {0,1} | X, el conjunto de los 365 dias del año$, entonces $c(x) = 1$ cuando en los datos nos dicen que con la instancia x conviene ir al cine, $c(x) = 0$ en caso que no. Por tanto para aprender la tarea T, necesitamos \emph{una hipotesis h en H tal que h(x) = c(x) para todas las x en X }. La tarea de aprendizaje del concepto $c$ requiere aprender el conjunto de instancias que lo satisface, describiendo este conjunto mediante una conjunción de restricciones sobre los atributos de la instancia.  

Estas hipotesis (funciones) pueden llegar a ser sumamente complejas y tener que mapear datos de entrada con muchas formas ej. imagenes, trayectorias, etc. En el caso de las redes neuronales, el espacio de hipótesis está determinado por la arquitectura de la red.
Vamos a definir el espacio de hipótesis, cuando decidamos qué neuronas vamos a poner en nuestro sistema, como las conectamos entre sí y cómo van a transferirse información de una a la otra y cuántas neuronas van a ser. Lo que veremos a lo largo del curso son diferentes arquitecturas y el impacto que tiene hacer diferentes modificaciones así como las matemáticas que existen detrás de estas. 

