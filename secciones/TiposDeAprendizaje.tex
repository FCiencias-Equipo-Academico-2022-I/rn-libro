\section{Tipos de aprendizaje}

\begin{description}
 \item [Aprendizaje supervisado] Aprender a predecir la salida, dado un vector de entrada.
 \begin{description}
  \item [Regresión] La salida está en $\mathbb{R}^n$.
  \item [Clasificación] La salida es discreta, usualmente etiquetas de clases.
 \end{description}
 
 \item [Aprendizaje no supervisado] Descubrir una buena representación para las entradas.  Una buena representación tiene las características siguientes:
 \begin{enumerate}
  \item Es compacta, es una representación en pocas dimensiones de entradas en varias dimensiones.
  \item Permite utilizar características que se pueden representar en forma económica (poco espacio).
 \end{enumerate}
 \begin{description}
  \item [Clasificación]
  \item [Análisis de componentes principales]
 \end{description}
 
 \item [Aprendizaje reforzado] Aprender a elegir una acción para maximizar una ganancia.
\end{description}