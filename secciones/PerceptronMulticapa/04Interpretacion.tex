\section{Interpretación matemática del mapeo no lineal}
En esta sección vamos a ver por qué agregar una capa más nos permite calcular funciones que no era posible calcular antes es decir no basta con tener más percepciones en realidad necesitamos tener percepciones que reciban como entrada la salida de percepciones anteriores y esto lo vamos a notar aquí por eso el selector es una función tan útil para poder explicar estos conceptos de entrada que teníamos al inicio de la función sort eran los valores 0 0 0 1 1 0 y 1 1 estos solamente los podría bueno puede utilizar perceptor únicamente para evaluar funciones que sean linealmente separables entonces cuando hemos aplicado la primera capa donde pudimos poner tantas operaciones como que hicimos lo que hicimos fue tomar estas entradas y mapear las a un conjunto distinto y esté ocurriendo en este nuevo conjunto 

Observen que las entradas ahora son 01 11 11 10 es decir estas dos están duplicadas ya no tengo el 0 0 por eso es que se dice que trabajamos con un mapeo no lineal nuestra función sigmoide o nuestra función escalón transformaron nuestro espacio de manera que no tenemos ahora una relación 1 a 1 qué efecto va a tener. Aquí tenemos una vez más cuáles son las entradas originales las entradas que se obtuvieron es bueno salidas de las capas ocultas que van a hacer ahora las entradas para el último perceptor entonces veamos acá qué pasó con este cuadro aquí teníamos cuatro datos que no eran linealmente separables después de la primera capa fue como haberlo plegado a otro espacio de dos dimensiones, pero observen ahora que como que esto es bueno vamos a ver quién se mató a quién el 0 0 se convirtió en 0,1 es decir este para acá el 0,1 se convirtió en 1,1 este fue para acá el 1010 también se convirtió en 1,1 es otro de acá y el 11 se convirtió en el 10 está aquí entonces nuestros puntos negros como que giraron un poquito y quedaron aquí y nuestros dos puntos blancos quedaron mapeados uno encima del otro en el mismo punto de este lado. 

Entonces ahora sí lo puedo separar con un plano qué pasa en medio de los dos cualquiera que lo logre es bueno entonces estos tres valores quedaron mapeados ahora a un nuevo espacio que por cierto tiene una dimensión solo hay un valor. Entonces el paso fundamental radicó en que pudimos mapear este espacio hacia un espacio nuevo donde si es posible separar a nuestros datos linealmente y sobre esto pues ya simplemente aplicamos lo que podía hacer cualquier perceptual matemáticamente ese es el poder que nos está añadiendo una capa de en medio y básicamente lo podemos generalizar ahora si a cualquier función. Es posible quitar los sesgos si utilizamos la función escalón como función de activación y definimos precisamente el menor o igual para la parte del escalón en esta ocasión vamos a obtener valores que están exactamente en este punto de salto así es que vamos a tener que decir quién queda a la izquierda quien quiere a la derecha y entonces podemos hacer esto igual se necesitan dos capas, pero podemos quitarnos la parte de los sesgos porque nuestras fronteras si están pasando por ser un coma se hace solo por curiosidad bien aquí tenemos entonces ya esa interpretación vemos que los ceros negativos se van a tomar como ceros y solamente los positivos van a quedar como unos y automáticamente se puede hacer esta versión resumida. 


