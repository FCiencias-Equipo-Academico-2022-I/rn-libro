\documentclass[12pt,openany]{book}
%\usepackage{classnotestikz}
%\usepackage{tikzelements}
\usepackage{libro-fciencias}
\usepackage{booktabs}
\usepackage{colortbl}
\def\thickline{\specialrule{.15em}{.05em}{.05em}}
\def\violetrule{\color{Violeta}{\rule{100px}{0.05em}}}
\def\bluerule{\color{DarkBlue}{\rule{100px}{0.05em}}}
\usepackage{multirow}


\usepackage{diagramas-fciencias}
\pgfplotsset{compat=1.15}

\graphicspath{ {Figuras/} }

%\setcounter{tocdepth}{4}

\addbibresource{rnnotesref.bib}


%----------------------------------------------------------------------------------------
%	Autores y Título
%----------------------------------------------------------------------------------------

\title{Redes Neuronales}
\subtitle{Notas de clase}
\author{Karla Fernanda Jiménez Gutiérrez\newline
        Verónica Esther Arriola Ríos}
\publisher{Facultad de Ciencias, UNAM}
\background{Neurona.png}


\begin{document}
\maketitle

%----------------------------------------------------------------------------------------
% Contenido
%----------------------------------------------------------------------------------------
\frontmatter % Numeración romana
\tableofcontents
\clearemptydoublepage % Whitespace to the end of the page


%----------------------------------------------------------------------------------------
%	                                Inicio
%----------------------------------------------------------------------------------------
\mainmatter  % Numeración arabiga


%%
\chapter*{Convenciones}

A lo largo del texto se utilizará la siguiente notación para diversos elementos:
\begin{longtable}{lc}
 Conjuntos   &   $\set{C}$ \\
 Vectores    &   $\vec{X}$ \\
 Matrices    &   $\mat{M}$ \\
 Unidades    &   $\unit{cm}$
\end{longtable}



%%
\part{Introducción}
\chapter{Neurona biológica}
\section{Sistema nervioso}
\section{Neurona biológica}
\section{Modelo de Hodgkin-Huxley: membrana y canal}
\section{Ecuaciones diferenciales}


%%
\part{Aún no tiene nombre} % === PARTE 2 || ===
\chapter{Tampoco tiene nombre pero es el dos} % === CAPITULO 2  ===
\section{Modelo de Hodgkin-Huxley. Dinámica de los disparos}
\section{Método de Euler}
\section{Hodgkin-Huxley en IPython notebook}

\chapter{II Aprendizaje de máquina} % === CAPITULO 3  === 
\section{Definición}
\section{Espacio de hipótesis}
\section{Conjuntos de entrenaiento, validación y prueba}
\section{Perceptrón}
\section{Compuertas lógicas con el perceptrón}
\section{Funciones de activación}
\section{Funciones de error: diferencias al cuadrado y entropía cruzada}
\section{Medidas de rendimiento: Matriz de confusión, precisión, recall, f score, etc}

\chapter{III Perceptrón multicapa} % === CAPITULO 4  === 
\section{XOR}
\section{Propagación hacia adelante manual}
\section{Propagación hacia adelante vectorizada (con matrices)}
\section{Expresividad de la hipótesis, dependencia de las neuronas en la capa de en medio}
\section{Teorema del aproximador universal (Michale Nielsen)}

\chapter{Entrenamiento} % === CAPITULO 5  ===
\section{Retropropagación. Gradiente de la función de error}
\section{Descenso por el gradiente}
\section{Otras funciones de optimización}

\chapter{Optimización del entrenamiento}% === CAPITULO 6  ===
\section{Redes Profundas}
\section{Gradiente desvaneciente (o que explota)}
\section{Entrenamiento en línea vs en lotes}
\section{Normalización y normalización por lotes}
\section{Regularización}

\chapter{V.Casos.Análisis e interpretación}% === CAPITULO 7  ===
\section{Red Hinton árbol familiar con numpy (entrenamiento)}
\section{Red Hinton árbol familiar con pytorch}

\chapter{----}% === CAPITULO 8  ===
\section{MNIST versión básica con numpy}
\section{VI. Neuroevolución}
\subsubsection{Entrenamiento con algoritmos genéticos}
\subsubsection{Neuroevolución profunda}

\chapter{VII Mapeos autoorganizados}% === CAPITULO 9  ===
\section{Mapeos autoo-organizados, Kohonen}

\chapter{VIII Redes Neuronales Convolucionales}% === CAPITULO 10  ===
\section{}

\chapter{VIII Redes Neuronales Recurrentes}% === CAPITULO 11  ===
\section{Parciales ordenadas (Werbos)}
\section{Intro RNR. Sistemas dinámicos}
\section{Red 3}
\section{Redes 1 y 2 para clasificación}
\section{Predicción de secuencias, hasta bidireccionales}
\section{Redes traductoras, hasta recursivas}
\section{Inicio cómputo de yacimiento} % esto esta entrecomillas

\chapter{IX Atención}% === CAPITULO 12  ===
\section{Casos de análisis de serie}
\section{LSTM}
\section{GRU}
\section{Aplicaciones: ejemplos de RNR con git de cvicom: etiquetado de palabras y conjugación de verbos}

\chapter{XI Redes de Hopfield}% === CAPITULO 13  ===
\section{Redes de hopfield}
\section{Máquinas de Boltzman}

\chapter{------------}% === CAPITULO 14  ===
\section{Entrenamiento}
\section{Partículas y partículas de fantasía}
\section{Máquinas de Boltzman Restringidas}

\chapter{XI Redes adversarias}% === CAPITULO 15  ===
\section{GANs}

%----------------------------------------------------------------------------------------
% Bibliografia
%----------------------------------------------------------------------------------------
\backmatter

\printbibliography[heading=bibintoc]

\end{document}
