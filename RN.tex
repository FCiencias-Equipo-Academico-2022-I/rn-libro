\documentclass[12pt,openany]{book}
%\usepackage{classnotestikz}
%\usepackage{tikzelements}
\usepackage{libro-fciencias}
\usepackage{booktabs}
\usepackage{colortbl}
\def\thickline{\specialrule{.15em}{.05em}{.05em}}
\def\violetrule{\color{Violeta}{\rule{100px}{0.05em}}}
\def\bluerule{\color{DarkBlue}{\rule{100px}{0.05em}}}
\usepackage{multirow}


\usepackage{diagramas-fciencias}
\pgfplotsset{compat=1.15}

\graphicspath{ {Figuras/} }

%\setcounter{tocdepth}{4}

\addbibresource{rnnotesref.bib}


%----------------------------------------------------------------------------------------
%	Autores y Título
%----------------------------------------------------------------------------------------

\title{Redes Neuronales}
\subtitle{Notas de clase}
\author{Karla Fernanda Jiménez Gutiérrez\newline
        Verónica Esther Arriola Ríos}
\publisher{Facultad de Ciencias, UNAM}
\background{Neurona.png}


\begin{document}
\maketitle

%----------------------------------------------------------------------------------------
% Contenido
%----------------------------------------------------------------------------------------
\frontmatter % Numeración romana
\tableofcontents
\clearemptydoublepage % Whitespace to the end of the page


%----------------------------------------------------------------------------------------
%	                                Inicio
%----------------------------------------------------------------------------------------
\mainmatter  % Numeración arabiga


%%
\chapter*{Etc}

A lo largo del texto se utilizará la siguiente notación para diversos elementos:
\begin{longtable}{lc}
 Conjuntos   &   $\set{C}$ \\
 Vectores    &   $\vec{X}$ \\
 Matrices    &   $\mat{M}$ \\
 Unidades    &   $\unit{cm}$
\end{longtable}



%%
\part{Antecedentes}
\chapter{Neurona biológica}
\section{Sistema nervioso}
\section{Neurona biológica}

\chapter{Modelo de Hodgkin-Huxley}
\section{Membrana y canal}
\section{Dinámica del voltaje durante un disparo}
\section{Simulación usando el método de Euler}

\chapter{Aprendizaje de máquina}
\section{Espacio de hipótesis}
\section{Conjuntos de entrenaiento, validación y prueba}
\section{Perceptrón}
\section{Compuertas lógicas con neuronas}
\section{Funciones de activación}
\section{Funciones de error: diferencias al cuadrado y entropía cruzada}
\section{Medidas de rendimiento:}
\subsection{Matriz de confusión}
\subsection{Precisión}
\subsection{Recall}
\subsection{f score}

\part{Redes dirigidas acíclicas}
\chapter{Perceptrón multicapa}
\section{XOR}
\section{Propagación hacia adelante manual}
\section{Propagación hacia adelante vectorizada (con matrices)}
\section{Interpretación matemática del mapeo no lineal}
\section{Propagación hacia adelante para el perceptrón multicapa}


%%
\chapter{Entrenamiento por retropropagación}
\section{Función de error}
\section{Gradiente de la función de error}
\section{Descenso por el gradiente}
\section{Otros algoritmos de optimización}

\chapter{Optimización del entrenamiento} 
\section{Problemas en redes profundas}
\section{Gradiente desvaneciente (o que explota) }
\section{Entrenamiento en línea vs en lotes}
\section{Normalización y normalización por lotes}
\section{Regularización}

\chapter{Caso de análisis e interpretación}

\section{Red Hinton árbol familiar con numpy (entrenamiento)}

\section{Red Hinton árbol familiar con pytorch}

\chapter{Entrenamiento con genéticos}
%\section{MNIST versión básica con numpy}
\section{Algoritmos genéticos}
\section{Neuroevolución}
\subsection{Antecedentes: Aprendizaje por refuerzo en videojuegos}
\subsection{Arquitectura para estimar la función de recompensa}
\subsection{Entrenamiento}

%\part{Aprendizaje no supervisado}
\chapter{Mapeos autoorganizados}
\section{Aprendizaje no supervisado}
\section{Mapeos autoo-organizados}
\section{Kohonen}

%%
%\part{NO TIENE NOMBRE}
\chapter{Redes Neuronales Convolucionales}
\section{Convolución}
\section{Redes Convolucionales}
\section{Softmax}
\section{MNIST}

%%
\part{Redes con ciclos}
\chapter{Redes Neuronales Recurrentes}
\section{Derivadas ordenadas}
\section{Retropropagación en el tiempo}
\section{Sistemas dinámicos y despliegue del grafo}
\section{Arquitectura recurrente universal}
\section{Función de error}
\section{Forzamiento del profesor}

\chapter{Atención}
%\section{Casos de análisis de serie}
\chapter{LSTM}
\chapter{GRU}
\chapter{Casos de análisis: etiquetado de palabras y conjugación de verbos}

\part{Redes no dirigidas}
\chapter{Redes de hopfield}
\section{Entrenamiento}

\chapter{Máquinas de Boltzman}
\section{Entrenamiento}
\subsection{Partículas y partículas de fantasía}
\subsection{Máquinas de Boltzman Restringidas}

\chapter{Redes adversarias}
\section{GANs}

\appendix 
\chapter{Ecuaciones diferenciales}

%----------------------------------------------------------------------------------------
% Bibliografia
%----------------------------------------------------------------------------------------
\backmatter

\printbibliography[heading=bibintoc]

\end{document}
